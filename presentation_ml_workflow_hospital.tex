\documentclass[aspectratio=169]{beamer}

\usepackage{booktabs}
\usepackage{graphicx}
\usepackage{tikz}
\usetikzlibrary{shapes.geometric, arrows, positioning, fit, backgrounds}
\usepackage{amsmath}
\usepackage{amssymb}

\usetheme{metropolis}

\title{Machine Learning Workflow in a Hospital Setting}
\subtitle{SDS6210 – Informatics for Health (Q4)}
\author{MSc Public Health Data Science}
\date{}

\begin{document}

\begin{frame}
    \titlepage
    \begin{center}
        \begin{tabular}{ll}
            \toprule
            \textbf{Student ID} & \textbf{Student Name} \\
            \midrule
            SDS6/46982/2024 & Cavin Otieno \\
            SDS6/47543/2024 & Laura Nabalayo Kundu \\
            SDS6/47659/2024 & John Andrew \\
            \bottomrule
        \end{tabular}
    \end{center}
\end{frame}

\section{Purpose of Machine Learning in Hospital Settings}

\begin{frame}{Why Machine Learning in Healthcare?}
    Machine learning represents a transformative paradigm in healthcare delivery, offering capabilities that augment traditional clinical decision-making processes. The integration of ML into hospital workflows addresses several critical challenges faced by modern healthcare systems.

    \begin{itemize}
        \item \textbf{Clinical Decision Support}: Real-time risk stratification and early warning systems for conditions such as sepsis, cardiac arrest, and deterioration.
        \item \textbf{Operational Efficiency}: Resource allocation optimization, patient flow management, and predictive scheduling.
        \item \textbf{Diagnostic Accuracy}: Enhanced image interpretation, pattern recognition in laboratory data, and differential diagnosis assistance.
        \item \textbf{Predictive Analytics}: Hospital readmission prediction, length of stay estimation, and population health management.
    \end{itemize}

    The fundamental shift involves moving from reactive medicine to proactive, personalized, and precision-based clinical interventions powered by data-driven insights.
\end{frame}

\begin{frame}{From Retrospective Analysis to Real-Time Clinical Applications}
    Traditional healthcare analytics focused on retrospective cohort studies and population-level statistics. Machine learning enables a paradigm shift toward real-time, individual-level predictions integrated directly into clinical workflows.

    \begin{itemize}
        \item \textbf{Retrospective Phase}: Secondary analysis of administrative claims, quality improvement initiatives, and research studies.
        \item \textbf{Proactive Phase}: Prospective risk modeling embedded within electronic health record systems.
        \item \textbf{Real-Time Phase}: Streaming analytics providing continuous risk scores and actionable alerts at the point of care.
    \end{itemize}

    \begin{block}{Key Distinction}
        The transition from descriptive and diagnostic analytics to predictive and prescriptive analytics represents the core value proposition of ML in modern hospital settings.
    \end{block}
\end{frame}

\section{Phase 1: Problem Definition and Clinical Question Formulation}

\begin{frame}{Defining the Clinical Problem}
    The foundation of any successful ML implementation in healthcare begins with a precisely formulated clinical question that aligns with organizational priorities and clinical needs.

    \begin{itemize}
        \item \textbf{Clinical Validity}: Does the outcome being predicted have clinical significance and actionable intervention options?
        \item \textbf{Feasibility Assessment}: Are the required data elements available, accessible, and of sufficient quality for model development?
        \item \textbf{Stakeholder Alignment}: Engagement with clinical champions, informaticists, and administrative leadership.
        \item \textbf{Success Metrics}: Clear definition of primary and secondary outcomes, including acceptable performance thresholds.
    \end{itemize}

    Common clinical applications include mortality prediction, readmission risk, sepsis early warning, deterioration detection, and length of stay estimation.
\end{frame}

\begin{frame}{Translating Clinical Questions to ML Objectives}
    Bridging the gap between clinical intuition and ML specifications requires careful consideration of outcome definitions, time horizons, and actionability.

    \begin{itemize}
        \item \textbf{Outcome Definition}: Precise operationalization of clinical endpoints (e.g., 30-day all-cause readmission vs. 30-day preventable readmission).
        \item \textbf{Prediction Window}: Determination of the relevant time horizon for clinical action (e.g., 24-hour, 48-hour, or 7-day predictions).
        \item \textbf{Threshold Selection}: Balance between sensitivity and specificity based on clinical consequences of false positives and false negatives.
    \end{itemize}
\end{frame}

\section{Phase 2: Data Collection and Preprocessing}

\begin{frame}{Data Sources in Hospital Settings}
    Hospital data ecosystems comprise multiple interconnected sources that must be harmonized for effective ML development.

    \begin{itemize}
        \item \textbf{Electronic Health Records (EHR)}: Demographics, vital signs, laboratory results, medications, procedures, and clinical notes.
        \item \textbf{Administrative Data}: Billing codes, diagnosis codes (ICD-10), procedure codes (CPT), and discharge summaries.
        \item \textbf{Real-Time Monitoring Data}: Bedside monitors, infusion pumps, ventilators, and other connected medical devices.
        \item \textbf{External Data Sources}: Social determinants of health, regional epidemiological data, and interoperability networks.
    \end{itemize}

    \begin{block}{Interoperability Standards}
        HL7 FHIR (Fast Healthcare Interoperability Resources) serves as the primary standard for healthcare data exchange, enabling seamless integration across disparate systems.
    \end{block}
\end{frame}

\begin{frame}{Data Quality and Preprocessing Challenges}
    Healthcare data presents unique preprocessing challenges that require domain expertise and sophisticated handling techniques.

    \begin{itemize}
        \item \textbf{Missing Data}: Systematic patterns of missingness (MNAR, MAR, MCAR) require appropriate imputation strategies or sensitivity analyses.
        \item \textbf{Temporal Heterogeneity}: Changes in clinical practices, coding habits, and documentation standards over time introduce distribution shifts.
        \item \textbf{Irregular Sampling}: Laboratory results and vital signs are collected at irregular intervals, necessitating time-series handling approaches.
        \item \textbf{Coding Variations}: ICD-10 code granularity, evolving code sets, and local coding practices require standardization.
    \end{itemize}

    Robust data validation pipelines and exploratory data analysis are essential prerequisites for model development.
\end{frame}

\section{Phase 3: Feature Engineering and Selection}

\begin{frame}{Feature Engineering in Clinical Domains}
    Feature engineering transforms raw clinical data into meaningful predictors that capture the underlying physiological or pathological processes relevant to the prediction task.

    \begin{itemize}
        \item \textbf{Temporal Aggregation}: Computing trends, rates of change, and temporal patterns from time-series data (e.g., delta neutrophil index, shock index).
        \item \textbf{Domain-Specific Transformations}: Composite scores such as SOFA (Sequential Organ Failure Assessment), NEWS (National Early Warning Score), and qSOFA.
        \item \textbf{Natural Language Processing}: Extraction of clinical entities, negation detection, and temporal reasoning from unstructured clinical notes.
        \item \textbf{Categorical Encodings}: Hierarchical groupings of diagnoses and procedures that capture clinical meaningful categories.
    \end{itemize}

    The choice of features should be guided by clinical plausibility and pathophysiological rationale.
\end{frame}

\begin{frame}{Feature Selection Strategies}
    Feature selection aims to identify the most predictive and generalizable subset of features while avoiding overfitting and maintaining model interpretability.

    \begin{itemize}
        \item \textbf{Filter Methods}: Correlation analysis, mutual information, chi-square tests, and variance thresholds for preliminary feature screening.
        \item \textbf{Wrapper Methods}: Recursive feature elimination (RFE) and forward/backward selection algorithms.
        \item \textbf{Embedded Methods}: L1 regularization (LASSO) and tree-based importance scores from random forests or gradient boosting.
        \item \textbf{Clinical Validation}: Feature review by subject matter experts to ensure clinical meaningfulness and data quality.
    \end{itemize}

    Balancing statistical significance with clinical interpretability is crucial for model adoption and regulatory compliance.
\end{frame}

\section{Phase 4: Model Selection and Training}

\begin{frame}{Model Selection Considerations}
    The selection of ML algorithms for healthcare applications requires careful consideration of the trade-offs between predictive performance, interpretability, and computational requirements.

    \begin{itemize}
        \item \textbf{Interpretable Models}: Logistic regression, decision trees, and score-based models that provide transparent decision pathways.
        \item \textbf{Black-Box Models}: Gradient boosting machines, random forests, and deep neural networks that offer superior predictive performance.
        \item \textbf{Time-to-Event Models}: Cox proportional hazards and survival analysis frameworks for time-to-event outcomes.
    \end{itemize}

    \begin{block}{The Interpretability-Performance Trade-off}
        While complex models may achieve higher discrimination, interpretable models facilitate clinical acceptance, regulatory review, and bias detection. The choice should align with the clinical use case and regulatory environment.
    \end{block}
\end{frame}

\begin{frame}{Model Training Best Practices}
    Rigorous model training protocols ensure reproducibility, generalizability, and fair algorithmic performance across patient populations.

    \begin{itemize}
        \item \textbf{Train-Validation-Test Split}: Temporal splits for time-series data to prevent data leakage and simulate real-world deployment.
        \textbf{Cross-Validation}: Stratified k-fold cross-validation for robust performance estimation, particularly with imbalanced outcomes.
        \item \textbf{Class Imbalance Handling}: SMOTE, class weighting, and threshold optimization for rare clinical events.
        \item \textbf{Hyperparameter Tuning}: Grid search, random search, or Bayesian optimization for model configuration.
    \end{itemize}
\end{frame}

\section{Phase 5: Model Validation and Performance Evaluation}

\begin{frame}{Validation Strategies}
    Comprehensive validation assesses model performance across multiple dimensions to ensure clinical utility and generalizability.

    \begin{itemize}
        \item \textbf{Internal Validation}: Bootstrap and cross-validation estimates of optimism-corrected performance metrics.
        \item \textbf{Temporal Validation}: Performance evaluation on holdout data from a later time period to assess temporal stability.
        \item \textbf{External Validation}: Testing on independent datasets from different institutions or healthcare systems.
        \item \textbf{Subgroup Validation}: Performance assessment across demographic subgroups to identify potential algorithmic bias.
    \end{itemize}
\end{frame}

\begin{frame}{Performance Metrics for Clinical Models}
    Evaluation metrics for healthcare ML models must align with clinical objectives and account for the consequences of prediction errors.

    \begin{itemize}
        \item \textbf{Discrimination}: Area under the receiver operating characteristic curve (AUC-ROC) for binary classification.
        \item \textbf{Calibration}: Hosmer-Lemeshow test, calibration curves, and Brier score to assess agreement between predicted and observed probabilities.
        \item \textbf{Clinical Utility}: Decision curve analysis and net benefit calculations to evaluate risk-benefit trade-offs.
        \item \textbf{Sensitivity and Specificity}: Performance at clinically relevant operating points, considering the costs of false positives and false negatives.
    \end{itemize}
\end{frame}

\section{Phase 6: Deployment and Clinical Integration}

\begin{frame}{Deployment Strategies}
    Transitioning from research prototypes to clinical systems requires careful architectural planning and phased implementation approaches.

    \begin{itemize}
        \item \textbf{Shadow Mode Deployment}: Running the model in parallel with clinical workflows without exposing predictions to end-users.
        \item \textbf{Pilot Implementation}: Limited deployment in specific clinical units or patient populations with enhanced monitoring.
        \item \textbf{Full Integration}: Production deployment within the electronic health record with user-facing interfaces and alert mechanisms.
    \end{itemize}

    Technical infrastructure must support real-time inference, high availability, and seamless integration with existing clinical systems.
\end{frame}

\begin{frame}{Integration into Clinical Workflows}
    Successful deployment depends on thoughtful integration that respects clinical workflows and minimizes alert fatigue.

    \begin{itemize}
        \item \textbf{User Interface Design}: In-context displays within clinician workflows, avoiding separate login requirements.
        \item \textbf{Alert Design}: Tiered alert severity, clear explanatory information, and actionable recommendations.
        \item \textbf{Feedback Mechanisms}: Capture of clinician acceptance or rejection of recommendations for continuous model improvement.
        \item \textbf{EHR Integration Standards}: Use of SMART on FHIR applications for standards-based, interoperable deployment.
    \end{itemize}
\end{frame}

\section{Model Governance, Monitoring, and Updating}

\begin{frame}{Model Governance Framework}
    Establishing robust governance structures ensures accountability, transparency, and responsible management of clinical AI systems.

    \begin{itemize}
        \item \textbf{Model Registry}: Documentation of model versions, training data, performance metrics, and deployment status.
        \item \textbf{Approval Workflows}: Multidisciplinary review committees including clinicians, data scientists, ethicists, and administrators.
        \item \textbf{Audit Trails}: Comprehensive logging of model predictions, user interactions, and clinical outcomes.
        \item \textbf{Incident Response}: Defined procedures for model-related adverse events and unexpected behaviors.
    \end{itemize}
\end{frame}

\begin{frame}{Model Monitoring and Drift Detection}
    Continuous monitoring is essential to detect performance degradation and maintain clinical safety over time.

    \begin{itemize}
        \item \textbf{Data Drift Detection}: Statistical monitoring of input feature distributions to identify population shifts.
        \item \textbf{Concept Drift Detection}: Tracking of outcome rates and prediction-actual differences over time.
        \item \textbf{Performance Monitoring}: Ongoing calculation of performance metrics against ground truth labels when available.
        \item \textbf{Alert Thresholding}: Statistical process control methods for detecting significant departures from baseline performance.
    \end{itemize}
\end{frame}

\begin{frame}{Model Updating and Retraining Strategies}
    Maintaining model relevance requires planned updating protocols that balance performance improvements against stability and validation requirements.

    \begin{itemize}
        \item \textbf{Scheduled Retraining}: Periodic model updates based on accumulated new data and performance benchmarks.
        \item \textbf{Triggered Retraining}: Updates initiated by detected performance degradation or significant population shifts.
        \item \textbf{Online Learning}: Continuous model adaptation to streaming data, with careful safeguards against catastrophic forgetting.
        \item \textbf{Version Control}: Careful management of model versions with full validation requirements for each update.
    \end{itemize}
\end{frame}

\section{Data Privacy, Ethics, and Regulatory Considerations}

\begin{frame}{Data Privacy Requirements}
    Healthcare ML development must comply with stringent privacy regulations and protect sensitive patient information throughout the data lifecycle.

    \begin{itemize}
        \item \textbf{Regulatory Compliance}: Adherence to HIPAA (US), GDPR (Europe), and local data protection frameworks.
        \item \textbf{De-identification}: Application of safe harbor or expert determination methods for protected health information.
        \item \textbf{Access Controls}: Role-based access, audit logging, and principle of least privilege for data access.
        \item \textbf{Consent Considerations}: Patient notification requirements and consent mechanisms for secondary use of clinical data.
    \end{itemize}
\end{frame}

\begin{frame}{Ethical Considerations in Clinical ML}
    Responsible development of clinical AI requires proactive attention to fairness, transparency, and accountability.

    \begin{itemize}
        \item \textbf{Algorithmic Fairness}: Assessment of model performance across demographic groups to identify and mitigate disparate impacts.
        \item \textbf{Bias Detection}: Analysis of training data for historical biases and systematic inequities in healthcare delivery.
        \item \textbf{Transparency}: Explainability features and clear communication about model limitations and uncertainty.
        \item \textbf{Accountability}: Clear lines of responsibility for model-related decisions and outcomes.
    \end{itemize}
\end{frame}

\begin{frame}{Regulatory Framework for Clinical AI}
    Clinical ML systems may be subject to regulatory oversight depending on their intended use and risk classification.

    \begin{itemize}
        \item \textbf{SaMD Classification}: Software as a Medical Device framework from FDA, EU MDR, and other regulatory bodies.
        \item \textbf{Risk-Based Validation}: Validation requirements scaled to the risk level of the clinical application.
        \item \textbf{Quality Management Systems}: FDA 21 CFR Part 820, ISO 13485, and other quality system requirements.
        \item \textbf{Post-Market Surveillance}: Ongoing monitoring requirements and adverse event reporting obligations.
    \end{itemize}
\end{frame}

\section{Clinical Decision Support Examples}

\begin{frame}{Example 1: Sepsis Early Warning System}
    Sepsis prediction represents a canonical use case for ML in hospital settings, demonstrating the full workflow from problem definition to deployment.

    \begin{itemize}
        \item \textbf{Problem}: Early identification of patients at risk of sepsis to enable timely antibiotic administration.
        \item \textbf{Data}: Vital signs, laboratory values (lactate, WBC, creatinine), and clinical notes.
        \item \textbf{Features}: SOFA score components, infection-related indicators, and temporal trends.
        \item \textbf{Model}: Gradient boosting classifier or random forest with interpretability enhancements.
        \item \textbf{Deployment}: Integrated into nursing station dashboards with tiered alert levels.
    \end{itemize}

    Clinical studies have demonstrated reduced mortality and improved outcomes with well-designed sepsis prediction systems.
\end{frame}

\begin{frame}{Example 2: Radiology Image Triage}
    Computer vision applications in radiology demonstrate the potential for ML to prioritize critical findings and reduce diagnostic delays.

    \begin{itemize}
        \item \textbf{Problem}: Prioritization of chest radiographs with potential acute findings (pneumothorax, pneumonia, effusion).
        \item \textbf{Data}: Historical chest radiographs with radiologist annotations and final reports.
        \item \textbf{Model}: Convolutional neural networks (CNN) for image classification and object detection.
        \item \textbf{Features}: Image embeddings, heatmaps highlighting regions of interest, and confidence scores.
        \item \textbf{Deployment}: Integration with PACS systems to flag studies requiring expedited radiologist review.
    \end{itemize}

    These systems augment radiologist workflows without replacing human interpretation, ensuring that critical findings receive timely attention.
\end{frame}

\section{Conclusion}

\begin{frame}{Key Takeaways}
    Successful implementation of ML in hospital settings requires attention to technical, organizational, and ethical dimensions throughout the model lifecycle.

    \begin{itemize}
        \item \textbf{Clinical Alignment}: ML solutions must address genuine clinical needs with actionable predictions and measurable outcomes.
        \item \textbf{Data Foundation}: Robust data quality and preprocessing are prerequisites for reliable model development.
        \item \textbf{Validation Rigor}: Comprehensive validation across temporal, external, and subgroup dimensions ensures generalizability.
        \item \textbf{Human-in-the-Loop}: Clinical ML should augment rather than replace human judgment, maintaining clinician agency.
    \end{itemize}
\end{frame}

\begin{frame}{Safety, Reliability, and Sustainability}
    The long-term success of clinical AI depends on maintaining safety standards, ensuring reliability, and building sustainable systems.

    \begin{itemize}
        \item \textbf{Safety First}: Continuous monitoring, fail-safe mechanisms, and rapid response protocols for adverse events.
        \item \textbf{Reliability Standards}: Consistent performance across patient populations, time periods, and clinical settings.
        \item \textbf{Sustainable Operations}: Dedicated resources for model maintenance, updating, and governance.
        \item \textbf{Organizational Learning**: Integration of ML insights into institutional knowledge and quality improvement processes.
    \end{itemize}

    \begin{block}{Final Message}
        Machine learning holds tremendous promise for improving healthcare outcomes, but realizing this potential requires responsible development, rigorous validation, and sustained commitment to patient safety and ethical principles.
    \end{block}
\end{frame}

\begin{frame}{References}
    \begin{itemize}
        \item Rajkomar, A., Oren, E., Chen, K., et al. (2018). Scalable and accurate deep learning with electronic health records. NPJ Digital Medicine, 1(1), 18.
        \item Sendak, M. P., Gao, M., Brajer, N., \& Balu, S. (2020). Presenting machine learning model information to clinical end users with model facts labels. NPJ Digital Medicine, 3(1), 41.
        \item Shah, N. H., Milstein, A., \& Bagley, S. C. (2019). Using machine learning to improve the value of care delivered to patients. JAMA, 322(22), 2155-2156.
        \item FDA. (2019). Software as a Medical Device (SaMD): Clinical Evaluation. FDA Guidance Document.
    \end{itemize}
\end{frame}

\end{document}
